\documentclass[12pt]{article}
\usepackage[a4paper, total={6in, 8in}]{geometry}

\setlength{\parskip}{3pt}

\usepackage{amsmath}
\usepackage{amssymb}
\usepackage{graphicx}
\usepackage{siunitx}
\usepackage{authblk}

\bibliographystyle{ieeetr}

\begin{document}

\title{Can liquid crystal phases be identified via machine learning?}
\author{Joshua Heaton}
\affil{School of Physics and Astronomy, The University of Manchester\\}
\date{\today}

\maketitle

\begin{abstract}
hi
\end{abstract}

\pagenumbering{gobble}
\newpage
\pagenumbering{arabic}

%=========================================================================================
\section{Introduction}
Machine learning methods have seen widespread utilisation across all scientific disciplines, in situations where conventional algorithms are too cumbersome to implement for specific data-based and modelling tasks. The range of possible applications was vastly increased with the advent of deep learning in 2006, which is loosely defined as machine learning with big datasets, parallel computation and complex, scalable algorithms. Convolutional neural networks (CNNs), a specific class of machine learning algorithms, were conceived in 1989 by Yann LeCun et al. and successfully applied to recognition of handwritten characters. However, their astounding performance in the field of computer vision would not be fully realised until the aforementioned deep learning revolution. Interest was further boosted when Alex Krizhevsky and Ilya Sutskever entered a CNN into the ImageNet Large Scale Visual Recognition Challenge in 2012, and won by a large margin.

Liquid crystal phases are in general identified by eye, directly from textures taken by polarised microscopy. This can prove a difficult task. Many unique phases, generated by often minor changes in structural properties, can have a similar textural appearance. Our project aims to test the viability of machine learning algorithms as tools to aid in this task. CNNs are particularly suitable due to their prevalence in image classification, and so form the core of our investigations. Current literature in this specific topic is limited, and the approaches so far have mostly involved the usage of simulated textures in the training of models \cite{Sigaki20, Minor20}. Sigaki et al. have demonstrated the viability of CNNs in isotropic and nematic phase texture classification and in the prediction of physical liquid crystal properties \cite{Sigaki20}. Our study further explores and attempts to push the limits of the classification task across a wider range of phases, utilising real experimental data produced by polarised microscopy.

This project report will first provide a brief overview of the physics behind liquid crystals and the capturing of their textures by polarised microscopy, as well as an introduction to machine learning, neural networks and CNNs. The details and results of our investigations into phase classification will then be presented, as well as an outlook to further study.
%=========================================================================================
\section{Liquid crystal phases}
a
\subsection{Physical structures}
a
\subsection{Capturing textures}
a
%=========================================================================================
\section{General machine learning principles}
a
\subsection{Tasks, performance measures and experience}
a
\subsection{Supervised and unsupervised learning}
a
\subsection{Optimisation}
a
\subsection{Underfitting and overfitting}
a
\subsection{Model capacity and regularisation}
a
\subsection{Hyperparamters}
a
%=========================================================================================
\section{Neural networks}
a
\subsection{Hidden units and network architecture}
a
\subsection{Neural network training}
a
\subsection{Regularisation methods}
a
%=========================================================================================
\section{Convolutional neural networks}
a
\subsection{Convolutional layers}
a
\subsection{Pooling layers}
a
%=========================================================================================
\section{4-phase classifier models}
a
\subsection{Dataset preparation}
a
\subsection{Model architectures and training configuration}
a
\subsection{Results}
a
%=========================================================================================
\section{Smectic phase classifier models}
a
\subsection{Dataset preparation}
a
\subsection{Model architectures and training configuration}
a
\subsection{Results}
a
%=========================================================================================
\section{Smectic A and C binary classifier models}
a
\subsection{Dataset preparation}
a
\subsection{Model architectures and training configuration}
a
\subsection{Results}
a
%=========================================================================================
\section{Conclusions}
a
%=========================================================================================
\section{Going forward}
a
%=========================================================================================

\bibliography{report}

\end{document}