\documentclass[12pt]{article}
\usepackage[a4paper, total={6in, 8in}]{geometry}

%\setlength{\parskip}{3pt}

\usepackage{amsmath}
\usepackage{amssymb}
\usepackage{graphicx}
\usepackage{siunitx}
\usepackage{authblk}
\usepackage{url}

\bibliographystyle{ieeetr}

\begin{document}

\title{Can liquid crystal phases be identified via machine learning?}
\author{Joshua Heaton}
\affil{School of Physics and Astronomy, The University of Manchester\\}
\date{\today}

\maketitle

\begin{abstract}
hi
\end{abstract}

\pagenumbering{gobble}
\newpage
\pagenumbering{arabic}

%=========================================================================================
\section{Introduction}
Machine learning methods have seen widespread utilisation across all scientific disciplines, in situations where conventional algorithms are too cumbersome to implement for specific data-based and modelling tasks \cite{Carleo19}. Deep learning, loosely defined as machine learning with large datasets, parallel computation and scalable algorithms with many layers \cite{Goodfellow16}, has and continues to increase the range and complexity of possible applications of machine learning in the sciences \cite{Carleo19}. Any task applying deep learning to data with a grid-like form, such as images, likely involves the usage of convolutional neural network (CNN) algorithms \cite{Goodfellow16}. CNNs were conceived in 1989 by Yann LeCun et al. and successfully applied to recognition of handwritten characters \cite{LeCun89}. However, their astounding performance in the field of computer vision would not be fully realised until after breakthroughs in deep learning starting in 2006 \cite{Goodfellow16}. Their efficacy was further proven when Geoffrey Hinton \textit{et al.} entered a CNN into the ImageNet Large Scale Visual Recognition Challenge in 2012, and won by a large margin \cite{ILSVRC15}.

Liquid crystal phases are in general identified by eye, directly from textures taken by polarised microscopy. Without adequate experience, this can prove a difficult task because certain unique liquid crystal phases, generated by often minor changes in structural properties, can have similar textural appearances \cite{Dierking03}. Our project aims to test the viability of machine learning algorithms as tools to assist phase identification. CNNs are particularly suitable due to their prevalence in image classification, and so form the core of our investigations. Current literature in this specific topic is limited, and the approaches so far have mostly involved the usage of simulated textures in the training of models \cite{Sigaki20, Minor20}. Sigaki et al. have demonstrated the viability of CNNs in isotropic and nematic phase texture classification and in the prediction of physical liquid crystal properties \cite{Sigaki20}. Our study further explores and attempts to push the limits of the classification task across a wider range of phases, utilising real experimental data produced by polarised microscopy.

This project report will first provide a brief overview of the physics behind liquid crystals and the capturing of their textures by polarised microscopy, as well as an introduction to machine learning, neural networks and CNNs. The details and results of our investigations into phase classification will then be presented, as well as an outlook to further study.
%=========================================================================================
\section{Liquid crystal phases}
Liquid crystals are substances in a state between that of a fully isotropic liquid and a crystal with a periodic lattice structure. The molecules can have varying positional order, and have orientational order over large sections. The unit vector parallel to the alignment of the molecules is called the director. Other details such as molecular shape and chirality affect the overall structure. These variations in structure result in numerous individual identifiable liquid crystal phases. Thermotropic liquid crystals exhibit phases transitions with changing temperature, whereas lyotropic liquid crystals are dissolved in a solvent with the phase depending on the concentration. This project will be concerned with only thermotropic liquid crystals. 

When cooling a thermotropic liquid crystal starting as an isotropic liquid, it will first transition to the nematic phase, which has orientational order only. The chiral nematic (cholesteric) phase also has no positional order, and has a periodic variation of the director, resulting in helical structures. Upon further cooling, the smectic phase will be reached. This can be split into three categories, going from fluid smectic to hexatic smectic to soft crystal in order of decreasing temperature. The fluid smectic phase has molecules arranged in layers, with no positional order in the plane of each layer. When the director is perpendicular to the layer planes, the phase is smectic A, with smectic C having a director that is tilted by comparison. Hexatic smectic phases have short range positional order within the layers in the form of hexagonal patterns, with

\subsection{Capturing textures}
%=========================================================================================
\section{General machine learning principles}
a
\subsection{Tasks, performance measures and experience}
a
\subsection{Supervised and unsupervised learning}
a
\subsection{Optimisation}
a
\subsection{Underfitting and overfitting}
a
\subsection{Model capacity and regularisation}
a
\subsection{Hyperparamters}
a
%=========================================================================================
\section{Neural networks}
a
\subsection{Hidden units and network architecture}
a
\subsection{Neural network training}
a
\subsection{Regularisation methods}
a
%=========================================================================================
\section{Convolutional neural networks}
a
\subsection{Convolutional layers}
a
\subsection{Pooling layers}
a
%=========================================================================================
\section{4-phase classifier models}
a
\subsection{Dataset preparation}
a
\subsection{Model architectures and training configuration}
a
\subsection{Results}
a
%=========================================================================================
\section{Smectic phase classifier models}
a
\subsection{Dataset preparation}
a
\subsection{Model architectures and training configuration}
a
\subsection{Results}
a
%=========================================================================================
\section{Smectic A and C binary classifier models}
a
\subsection{Dataset preparation}
a
\subsection{Model architectures and training configuration}
a
\subsection{Results}
a
%=========================================================================================
\section{Conclusions}
a
%=========================================================================================
\section{Going forward}
a
%=========================================================================================

\bibliography{report}

\end{document}