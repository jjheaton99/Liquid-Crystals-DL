\documentclass[12pt]{article}
\usepackage[a4paper, total={6.6in, 9.4in}]{geometry}

%\setlength{\parskip}{3pt}

\usepackage{amsmath}
\usepackage{amssymb}
\usepackage{bm}
\usepackage{graphicx}
\usepackage{siunitx}
\usepackage{authblk}
\usepackage{url}
\usepackage{appendix}
\usepackage{booktabs}
\usepackage[font=small]{caption}
\usepackage{subcaption}

\usepackage [english]{babel}
\usepackage [autostyle, english = american]{csquotes}
\MakeOuterQuote{"}

\bibliographystyle{ieeetr}

\begin{document}

\title{Multi-phase classification of liquid crystal textures using convolutional neural networks}
\author{\textit{Joshua Heaton}\\\textit{10133722}}
\affil{Department of Physics and Astronomy, The University of Manchester}
\affil{MPhys project report}
\affil{Project performed in collaboration with James Harbon\\Supervisor: Dr Ingo Dierking}
\date{\today}

\maketitle

\begin{abstract}
\end{abstract}

\pagenumbering{gobble}
\newpage
\tableofcontents

\pagenumbering{gobble}
\newpage
\pagenumbering{arabic}

%~~~~~~~~~~~~~~~~~~~~~~~~~~~~~~~~~~~~~~~~~~~~~~~~~~~~~~~~~~~~~~~~~~~~~~~~~~~~~~~~~~~~~~~~~
\section{Introduction}
Machine learning (ML) is the term assigned to a wide range of computer algorithms that use data to automatically improve their performance on a specific task. These tasks can take various forms, including decision making, pattern recognition, and prediction  \cite{Murphy12}. A sub-field of ML known as deep learning (DL) generally consists of applying large-scale multi-layer neural networks, a type of algorithm inspired by the structure of the brain, to tasks involving highly complex abstractions of data. Such intensive algorithms typically require vast quantities of data and powerful computational resources to be trained effectively, with the advantage that they do not require any manual feature extraction \cite{Goodfellow16}. With the recent explosion in availability of such data and sophisticated computing technology, DL has seen a surge in interest and application among several fields \cite{Shrestha19}. Computer vision is one such field that has been impacted greatly. Convolutional neural networks (CNNs), a type of neural network suited particularly well to grid-based data, have proven extremely successful in the tasks of image classification, segmentation, and object detection \cite{Voulodimos18}.

There are many thousands of individual documented liquid crystal (LC) compounds, with each displaying a certain sequence of identifiable phases between that of a liquid and solid \cite{Dierking03}. Commonly, polarised microscopy is used to capture images of the textures produced by LC phases for identification by eye \cite{Dierking03}. Literature on machine learning for LC phase classification is sparse, with most studies focusing on the extraction of physical properties of LCs using simulated texture data \cite{Sigaki20, Sigaki19, Minor20, Walters19}, or other means \cite{Florin07, Butnariu13, Doi19, Inokuchi20}. Of most relevance is the work by Sigaki et al., in which they utilise CNNs to classify simulated isotropic and nematic phases to high accuracy \cite{Sigaki20}.

In this project, we prepare a novel dataset of LC texture images captured by polarised microscopy (PM), spanning multiple phases of all orders. Subsequently, we apply CNN classifier models to various phase groupings, probing the limits of attainable model accuracy. The work expands on and consolidates that of the first semester report \cite{Heaton20}, in which we demonstrated the viability of CNNs in some simple LC phase classification tasks. This report will provide a brief overview of LC phases, supervised ML, and CNNs, with further detail found in the first report \cite{Heaton20}. Details of the models used will then be provided, followed by a presentation of the results when they are applied to each of the prepared datasets. Summary conclusions and the limitations of the study will then be discussed.
%~~~~~~~~~~~~~~~~~~~~~~~~~~~~~~~~~~~~~~~~~~~~~~~~~~~~~~~~~~~~~~~~~~~~~~~~~~~~~~~~~~~~~~~~~
\section{Background principles}

\subsection{Liquid crystals}
Liquid crystal phases are characterised by the positional and orientational order of the molecular arrangement. In general, the phase of lyotropic LCs depends on the concentration of the sample in a solvent whilst thermotropic LCs, studied in this project, become more ordered with decreasing temperature. The order and overall structure of the LC phase determines its optical properties, in particular its birefringence. This enables images of the textures of a liquid crystal to be obtained by polarised microscopy, in which the sample is placed between two perpendicularly aligned polarisers. Polarised light incident on the set-up is altered in accordance with the current phase of the LC sample, producing characteristic features in the resulting image.

At sufficiently high temperatures, thermotropic LCs take the form of a fully anisotropic liquid with no structural order and hence birefringence, resulting in completely dark PM textures. Upon cooling, they will display at least one higher ordered phase before reaching the fully crystalline stage. Of lowest order, just orientational, is the nematic (N) phase, in which the molecules are aligned along a particular axis, called the director, and are still free to move around as in a liquid. Compounds with chiral molecules may instead display the cholesteric (N*) phase, which is the same as the nematic phase except with helical variation of the director. Layered positional order is introduced in the smectic phase, which is divided into three distinct phase groupings. The orientation of the director further categorises these groupings. The fluid smectic (FSm) phases have no positional order within molecular layers. The orientation of the director with respect to the layer planes determines whether the phase is smectic A (SmA), in which it is perpendicular, or C (SmC) otherwise. The smectic B (SmB), I (SmI), and F (SmF) phases are placed into the hexatic smectic (HSm) group, with short-range hexagonal structures generating positional order within the layers. The soft crystal phases differ in that the layers show long-range positional order.

This project uses CNNs to classify PM textures from thermotropic chiral LC compounds, including the phases N*, SmA, SmC, SmI, and SmF.  

\subsection{A brief overview of supervised machine learning}
Machine learning algorithms can in general be categorised as supervised, unsupervised, or reinforcement learning. The ML implementations of this project are purely supervised learning algorithms. In this case, the ML model is defined as a function, parametrised by learned values $\bm{\theta}$, that maps an input data sample $\bm{x}$ to an output predicted label $\hat{\bm{y}}$,
\begin{equation}
\hat{\bm{y}}=f(\bm{x};\bm{\theta}). \label{supmodel} 
\end{equation}
$\hat{\bm{y}}$ can take various forms depending on the specific task, for example regression, in which the model attempts to predict a continuous value given the input data, or classification, in which it predicts a category that the input data sample belongs to. A supervised model attempts to learn appropriate $\bm{\theta}$ values for the mapping using a set of training data, consisting of pairs, $i$, of input examples and their corresponding true labels, $\lbrace\bm{x}^{(i)},\bm{y}^{(i)}\rbrace$. The model takes sample $\bm{x}^{(i)}$ and produces an output label prediction $\hat{\bm{y}}^{(i)}$ according to Equation \ref{supmodel}. A chosen cost function $J(\bm{y},\hat{\bm{y}};\bm{\theta})$ is then evaluated, which generally provides a measure of the divergence of the model outputs from the true labels. The parameters are then updated with a particular optimisation algorithm in order to minimise the cost. Successful training as such results in a model that is effective in producing accurate predictions for new unseen data samples.



\subsection{A brief overview of convolutional neural networks}

%~~~~~~~~~~~~~~~~~~~~~~~~~~~~~~~~~~~~~~~~~~~~~~~~~~~~~~~~~~~~~~~~~~~~~~~~~~~~~~~~~~~~~~~~~
\section{Summary of previous work}

%~~~~~~~~~~~~~~~~~~~~~~~~~~~~~~~~~~~~~~~~~~~~~~~~~~~~~~~~~~~~~~~~~~~~~~~~~~~~~~~~~~~~~~~~~
\section{CNN Models}

\subsection{Sequential}

\subsection{Inception}

\subsection{ResNet}

%~~~~~~~~~~~~~~~~~~~~~~~~~~~~~~~~~~~~~~~~~~~~~~~~~~~~~~~~~~~~~~~~~~~~~~~~~~~~~~~~~~~~~~~~~
\section{Methodology}

\subsection{Data preparation}

\subsection{Model training configurations}

%~~~~~~~~~~~~~~~~~~~~~~~~~~~~~~~~~~~~~~~~~~~~~~~~~~~~~~~~~~~~~~~~~~~~~~~~~~~~~~~~~~~~~~~~~
\section{Classification tasks and results}

\subsection{Cholesteric and smectic}

\subsection{Smectic A and C}

\subsection{Smectic I and F}

\subsection{Cholesteric, fluid smectic and hexatic smectic}

\subsection{Cholesteric, smectic A, C, and hexatic smectic}

%~~~~~~~~~~~~~~~~~~~~~~~~~~~~~~~~~~~~~~~~~~~~~~~~~~~~~~~~~~~~~~~~~~~~~~~~~~~~~~~~~~~~~~~~~
\section{Conclusions}

%~~~~~~~~~~~~~~~~~~~~~~~~~~~~~~~~~~~~~~~~~~~~~~~~~~~~~~~~~~~~~~~~~~~~~~~~~~~~~~~~~~~~~~~~~
\bibliography{report2}

\appendix
\appendixpage

\end{document}