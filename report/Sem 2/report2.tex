\documentclass[12pt]{article}
\usepackage[a4paper, total={6.6in, 9.4in}]{geometry}

%\setlength{\parskip}{3pt}

\usepackage{amsmath}
\usepackage{amssymb}
\usepackage{bm}
\usepackage{graphicx}
\usepackage{siunitx}
\usepackage{authblk}
\usepackage{url}
\usepackage{appendix}
\usepackage{booktabs}
\usepackage[font=small]{caption}
\usepackage{subcaption}

\usepackage [english]{babel}
\usepackage [autostyle, english = american]{csquotes}
\MakeOuterQuote{"}

\bibliographystyle{ieeetr}

\begin{document}

\title{Multi-phase classification of liquid crystal textures using convolutional neural networks}
\author{\textit{Joshua Heaton}\\\textit{10133722}}
\affil{Department of Physics and Astronomy, The University of Manchester}
\affil{MPhys project report}
\affil{Project performed in collaboration with James Harbon\\Supervisor: Dr Ingo Dierking}
\date{\today}

\maketitle

\begin{abstract}
\end{abstract}

\pagenumbering{gobble}
\newpage
\tableofcontents

\pagenumbering{gobble}
\newpage
\pagenumbering{arabic}

%~~~~~~~~~~~~~~~~~~~~~~~~~~~~~~~~~~~~~~~~~~~~~~~~~~~~~~~~~~~~~~~~~~~~~~~~~~~~~~~~~~~~~~~~~
\section{Introduction}
Machine learning (ML) is the term assigned to a wide range of computer algorithms that use data to automatically improve their performance on a specific task. These tasks can take various forms, including decision making, pattern recognition, and prediction  \cite{Murphy12}. A sub-field of ML known as deep learning (DL) generally consists of applying large-scale multi-layer neural networks, a type of algorithm inspired by the structure of the brain, to tasks involving highly complex abstractions of data. Such intensive algorithms typically require vast quantities of data and powerful computational resources to be trained effectively, with the advantage that they do not require any manual feature extraction \cite{Goodfellow16}. With the recent explosion in availability of such data and sophisticated computing technology, DL has seen a surge in interest and application among several fields \cite{Shrestha19}. Computer vision is one such field that has been impacted greatly. Convolutional neural networks (CNNs), a type of neural network suited particularly well to grid-based data, have proven extremely successful in the tasks of image classification, segmentation, and object detection \cite{Voulodimos18}.

There are many thousands of individual documented liquid crystal (LC) compounds, with each displaying a certain sequence of identifiable phases between that of a liquid and solid \cite{Dierking03}. Commonly, polarised microscopy is used to capture images of the textures produced by LC phases for identification by eye \cite{Dierking03}. Literature on machine learning for LC phase classification is sparse, with most studies focusing on the extraction of physical properties of LCs using simulated texture data \cite{Sigaki20, Sigaki19, Minor20, Walters19}, or other means \cite{Florin07, Butnariu13, Doi19, Inokuchi20}. Of most relevance is the work by Sigaki et al., in which they utilise CNNs to classify simulated isotropic and nematic phases to high accuracy \cite{Sigaki20}.

In this project, we prepare a novel dataset of LC texture images captured by polarised microscopy, spanning multiple phases of all orders. Subsequently, we apply CNN classifier models to various phase groupings, probing the limits of attainable model accuracy. The work expands on and consolidates that of the first semester report \cite{Heaton20}, in which we demonstrated the viability of CNNs in some simple LC phase classification tasks. This report will provide a brief overview of LC phases, supervised ML, and CNNs, with further detail found in the first report \cite{Heaton20}. Details of the models used will then be provided, followed by a presentation of the results when they are applied to each of the prepared datasets. Summary conclusions and the limitations of the study will then be discussed.
%~~~~~~~~~~~~~~~~~~~~~~~~~~~~~~~~~~~~~~~~~~~~~~~~~~~~~~~~~~~~~~~~~~~~~~~~~~~~~~~~~~~~~~~~~
\section{Background principles}

\subsection{Liquid crystals}

\subsection{A brief overview of supervised machine learning}

\subsection{A brief overview of convolutional neural networks}

%~~~~~~~~~~~~~~~~~~~~~~~~~~~~~~~~~~~~~~~~~~~~~~~~~~~~~~~~~~~~~~~~~~~~~~~~~~~~~~~~~~~~~~~~~
\section{Summary of previous work}

%~~~~~~~~~~~~~~~~~~~~~~~~~~~~~~~~~~~~~~~~~~~~~~~~~~~~~~~~~~~~~~~~~~~~~~~~~~~~~~~~~~~~~~~~~
\section{CNN Models}

\subsection{Sequential}

\subsection{Inception}

\subsection{ResNet}

%~~~~~~~~~~~~~~~~~~~~~~~~~~~~~~~~~~~~~~~~~~~~~~~~~~~~~~~~~~~~~~~~~~~~~~~~~~~~~~~~~~~~~~~~~
\section{Methodology}

\subsection{Data preparation}

\subsection{Model training configurations}

%~~~~~~~~~~~~~~~~~~~~~~~~~~~~~~~~~~~~~~~~~~~~~~~~~~~~~~~~~~~~~~~~~~~~~~~~~~~~~~~~~~~~~~~~~
\section{Classification tasks and results}

\subsection{Cholesteric and smectic}

\subsection{Smectic A and C}

\subsection{Smectic I and F}

\subsection{Cholesteric, fluid smectic and hexatic smectic}

\subsection{Cholesteric, smectic A, C, and hexatic smectic}

%~~~~~~~~~~~~~~~~~~~~~~~~~~~~~~~~~~~~~~~~~~~~~~~~~~~~~~~~~~~~~~~~~~~~~~~~~~~~~~~~~~~~~~~~~
\section{Conclusions}

%~~~~~~~~~~~~~~~~~~~~~~~~~~~~~~~~~~~~~~~~~~~~~~~~~~~~~~~~~~~~~~~~~~~~~~~~~~~~~~~~~~~~~~~~~
\bibliography{report2}

\appendix
\appendixpage

\end{document}